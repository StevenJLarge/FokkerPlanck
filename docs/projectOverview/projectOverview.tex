\documentclass[15pt]{article}

\usepackage{amsmath}
\usepackage{amssymb}
\usepackage{amsfonts}
\usepackage{graphicx}
\usepackage{setspace}
\usepackage{tikz}
\usetikzlibrary{backgrounds}

\usepackage{fancyhdr}
\usepackage{geometry}
\geometry{margin=1in}
\usepackage{pgfplots}

%\usepackage[comma,super,sort&compress]{natbib}
\usepackage{natbib}
\setlength{\bibsep}{3pt plus 0.5ex}
\bibliographystyle{unsrt}

\usepackage[colorlinks=true,linkcolor=blue,urlcolor=blue,citecolor=blue]{hyperref}

\pagestyle{fancy}
\fancyhf{}
\rhead{\textsf{\thepage}}
\lhead{\textsf{\thesection}}
\cfoot{\texttt{Steven Large}}

\usepackage{tikz}
\usepackage{eso-pic}

\renewcommand{\headrulewidth}{0.4pt}
\renewcommand{\footrulewidth}{0.4pt}

\newcommand{\bs}[1]{\boldsymbol{#1}}
\newcommand{\mc}[1]{\mathcal{#1}}

\title{\sf Fokker-Planck equation: Numerical solutions and integration}
\author{\sf Steven Large}

\begin{document}

\maketitle
\tableofcontents
\pagebreak

\section{Introduction}

At an ensemble level, the dynamics of a particular stochastic process can be captured by equations that govern the time-dependent evolution ofits probability distribution over system states. For instance, systems which are described microscopically by stochatic differential equations (SDEs)---like the Langevin equation---can be analagously described at the level of a probability distribution through solving a partial differential equation (PDE).  In fact, one can show this formally by deriving the equations of motion for the trajectory- or distribution-level dynamics of a general stochastic system through the Chapman-Kolmogorov equation. Generally speaking, the mathematical description of such a system---as codified, for instance, by its L\'{e}vy-Khintchine decomposition---is captured by three distinct contributions: drift, brownian diffusion, and jumps. The standard Langevin equation captured the contributions of drift and Brownian diffusion, while the presence of jumps leads to more exotic stochastic processes---and corresponding mathematical descriptions---through, for instance, L\'{e}vy processes.\footnote{A tractible and familiar example of this in the world of physics would be anomalous diffusion---either super- or sub-diffusive dynamics.} In the absence of jumps, modelling becomes mroe simple, and while the trajectory-level dynamics are goverened by a Langevin equation, the time-dependent evolution of the ensemble-level probability distribution is govenred by a second-order parabolic PDE known as the Fokker-Planck equation.\footnote{In the presence of jumps, the analogous PDE is muchn more difficult to handle, and will often involve fractional derivative terms.}

From a practical standpoint, while trajectory-level dynamics can be relatively simple to understand, and nearly-trivial to simulate numerically, obtaining statistical quantitites can often be numerically expensive and time-consuming, with extra care being necessary to ensure that a sufficiently representative sample of trajectories have been sampled so that the statistical quantities being calculated have convereged.  Put another way, when inferring properties of a stochastic system from trajectory simulations, one has to assume that the entire trajectory distribution has been adequately sampled. This is not, however, a problem for direct solutions to the corresponding PDE, as the solution itself gives all information about the distirbution of states.

Mathematically, the Fokker-Planck equation can be represented by the general flux-conservative form
\begin{equation}
    \partial_t P(x, t) = \partial_x J(x, t)
\end{equation}
where $p(x, t)$ is the space- and time-dependent probability density function of a system, and $J(x, t)$ is the instantaneous probability flux through location $x$ at time $t$.x


% Introduce Fokker-Planck equation as an important equation
The Fokker-Planck equation is a tremendously important relation in the study of all types of stochastic systems. Starting from the most basic `continuity equation' of stochastic processes---the Chapman-Kolmogorov equation---one can show that, under a set of reasonable assumptions on the continuity and smoothness of the process itself, the governing dynamics at the level of a probability distribution is the Fokker-Planck equation.  Put simply, the Fokker-Planck equation takes the form of a 2nd order parabolic partial differential equation, describing the time-dependent evolution of a probability distribution (among other things).
% Superset of other important equations


\begin{equation}
    \partial_t p(\bs{x}) = -\sum_{i=1}^{N}\partial_{x_i}\left[ \mu(\bs{x}, t) p(\bs{x}, t) \right] + \sum_{i,j = 1}^N \partial_{x_i, x_j}^2 \left[ D_{ij}(\bs{x}, t) p(\bs{x}, t) \right] \label{eq:fokker-planck-general}
\end{equation}

\begin{equation}
    \partial_t p(x, t) = -\partial_x \left[ \mu(x, t)p(x, t) \right] + D\partial_{xx}^2p(x, t) \label{eq:fokker-planck-1D}
\end{equation}

\begin{equation}
    \partial_t p(x, t) = \beta D \partial_x\left[ \partial_x E(x, t) p(x, t) \right] + D\partial_{xx}^2 p(x, t) \label{eq:smoluchowski}
\end{equation}

\part{Numerical Integration}


\section{Numerical integration: general properties}
% Explicit and Implicit integration routine and discussion of general properties of numerical integration
% Introduce general discretization schemes
The central difficulty in solving PDEs is finding an accurate discrete representation of the function and its evolution so that the continuous-space aspects are preserved.  While this may naively seem like a relatively straightforward task, there is a tremendous subtlety to the particular way in which the equations are discretized, with many possible techniques, each having its own benefits and drawbacks.

For the case fo simplicity and concreteness, we will focus on a PDE with one spatial dimension for the time being, and will generalize to multiple spatial dimensions later on. For such a system, the PDE go erning the system dynamics is

\section{Stability}

% Discuss integrator splitting here as a means of segregating the simplifying stability analysis if different terms in the equation.

\section{Advection}
% Advection terms
To start, we focus on the \textit{advection} term in the Fokker-Planck equation [Eq.~\eqref{eq:fokker-planck-1D}]:
\begin{equation}
    \partial_t p(x, t) = -\partial_x \left[ \mu(x, t) p(x, t) \right] \label{eq:advection-general} \ .
\end{equation}
In this case, there is no diffusive component to the motion.  For integrating this component of the dynamics, we first introduce the Lax method. In itself, the Lax method is a relatively straightforward modification to the standard Euler scheme, approximating the spatial derivative with a centered difference and the temporal derivative with a form of centered difference utilizing the centered average $\tfrac{1}{2}(p_{i-1}^n + p_{i+1}^n)$ in place of the $p_i^n$ term in the usal form of the Euler scheme:
\begin{subequations}
\begin{align}
    \partial_t p(x, t) \quad &\to \quad \frac{1}{\Delta t}\left[ p_i^{n+1} - \frac{1}{2}\left( p_{i-1}^n + p_{i+1}^n \right)\right] \label{eq:lax-time-deriv} \\
    \partial_x\left[ \mu(x, t)p(x, t) \right] \quad &\to \quad \frac{1}{2\Delta x}\left( F_{i+1}^n + F_{i-1}^n \right) \label{eq:lax-space-deriv}
\end{align}
\end{subequations}
where $F_{i}^n \equiv \mu_i^np_i^n$ is the flux throguh state $i$ at time $n$.  While the Lax method's seemingly inconsequential replacement of $p_i^n$ in the temporal derivative with its centered average turns out to remedy a large number of the numerical instabilities found in the straightforward Euler discretization scheme.

More concretely, while the Euler scheme applied to advective equations of the form in Eq.~\eqref{eq:advection-general} will be \emph{unconditionally unstable} in the Neumann analysis sense (see Appendix~\ref{app:neumann-analysis}), a problem which is not present in the Lax method. Specifically, one can show that the discretization scheme will be stable so long as the inequality
\begin{equation}
    \frac{\max|\mu(x, t)|\Delta t}{\Delta x} \leq 1 \label{eq:lax-cfl}
\end{equation}
is satisfied. Eq.~\eqref{eq:lax-cfl} is an example of a \emph{Courant-Fredrichs-Lewy} (or CFL) cirterion, which is the condition that must be met for Neumann stability to hold.  However, even if the CFL criteria is met, the system will still exhibit inaccuracy due to the phenomenon of numerical dissipation. For the sake on concreteness, we can write the Lax approximation to the advection equation as the update equation
\begin{equation}
    p_i^{n+1} = \frac{1}{2}\left( p_{i+1}^n + p_{i-1}^n \right) - \frac{\Delta t}{2\Delta x}\left( F_{i+1}^n - F_{i-1}^n \right) \label{eq:lax-explicity}
\end{equation}
which can be re-ordered to mirror the structure of the FTCS approximation as
\begin{equation}
    \frac{p_{i}^{n+1} - p_{i}^{n}}{\Delta t} = \frac{-1}{2\Delta x}\left( F_{i+1}^n - F_{i-1}^n \right) + \frac{1}{2\Delta t}\left( p_{i+1}^n - 2p_i^n + p_{i-1}^n  \right) \ ,
\end{equation}
which is exactly the FCTS discretization for an advection-diffusion equation. Thus, the differencing scheme, while stable, represents the numerical evolution of an equation with a diffusive term--even though the equation we are trying to represent [Eq.~\eqref{eq:advection-general}] contains no such term. For this reason, the Lax method is said to display numerical dissipation, which can lead to inaccuracies in the solutions.

Given that, for our purposes, this numerical dissipation is worrisome, we use instead a modification of the Lax method---known as the two-step Lax-Wendroff method---which cures the numerical dissipation of the Lax method while maining the same stability. Here, the modification comes through defining half-step fluxes so that we can update the probability $p_i^n$ using the standard FTCS method with half-time and half-space qualtities
\begin{equation}
    p_i^{n+1} = p_i^n + \frac{\Delta t}{\Delta x}\left( F_{i+1/2}^{n+1/2} - F_{i-1/2}^{n+1/2} \right) \label{eq:lw-step-2}
\end{equation}
where the half-step fluxes are determined through the Lax method:
\begin{equation}
    p_{i+1/2}^{n+1/2} = \frac{1}{2}\left( p_{i+1}^n + p_{i}^n \right) - \frac{\Delta t}{\Delta x}\left( F_{i+1}^n - F_i^n \right) \label{eq:lw-step-1}
\end{equation}
and---as before---the half-step fluxes in Eq.\eqref{eq:lw-step-2} are calcualted as $F_{i\pm1/2}^{n+1/2} \equiv \mu_{i\pm1/2}^{n+1/2}p_{i\pm1/2}^{n+1/2}$.

\section{Diffusion}
    % Diffusion terms

\section{Fokker-Planck}


\section{Driven Harmonic Trap}

    \subsection{Excess Work in a Translating Track}

    \subsection{Excess Work in a Breathing Trap}


    \section{Driven Rotary Machine}

    \subsection{Steady-State Flux}

    \subsection{Precision and Driving Accuracy}

    % Other subsections?
    \section{Information Erasure}

    \section{Kolmogorov Backwards Equation}

    \subsection{First passage times}

    \subsection{Financial mathematics: Options pricing}


\part{Steady-state solutions}

% Work through this section on relaxation methods


\appendix

\section{\label{app:neumann}Neumann stability analysis}

\subsection{FCTS Scheme: advection and diffusion}

\subsection{Advection: Lax scheme}

\subsection{Advection: Lax-Wendroff scheme}

\subsection{Diffusion: Crank-Nicolson Scheme}

\end{document}