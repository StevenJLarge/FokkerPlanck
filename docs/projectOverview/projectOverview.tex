\documentclass[15pt]{article}

\usepackage{amsmath}
\usepackage{amssymb}
\usepackage{amsfonts}
\usepackage{graphicx}
\usepackage{setspace}
\usepackage{tikz}
\usetikzlibrary{backgrounds}

\usepackage{fancyhdr}
\usepackage{geometry}
\geometry{margin=1in}
\usepackage{pgfplots}

%\usepackage[comma,super,sort&compress]{natbib}
\usepackage{natbib}
\setlength{\bibsep}{3pt plus 0.5ex}
\bibliographystyle{unsrt}

\usepackage[colorlinks=true,linkcolor=blue,urlcolor=blue,citecolor=blue]{hyperref}

\pagestyle{fancy}
\fancyhf{}
\rhead{\textsf{\thepage}}
\lhead{\textsf{\thesection}}
\cfoot{\texttt{Steven Large}}

\usepackage{tikz}
\usepackage{eso-pic}

\renewcommand{\headrulewidth}{0.4pt}
\renewcommand{\footrulewidth}{0.4pt}

\newcommand{\bs}[1]{\boldsymbol{#1}}
\newcommand{\mc}[1]{\mathcal{#1}}

\title{\sf Fokker-Planck equation: Numerical solutions and integration}
\author{\sf Steven Large}

\begin{document}

\maketitle
\tableofcontents
\pagebreak

\section{Introduction}

At an ensemble level, the dynamics of a particular stochastic process can be captured by equations that govern the time-dependent evolution ofits probability distribution over system states. For instance, systems which are described microscopically by stochatic differential equations (SDEs)---like the Langevin equation---can be analagously described at the level of a probability distribution through solving a partial differential equation (PDE).  In fact, one can show this formally by deriving the equations of motion for the trajectory- or distribution-level dynamics of a general stochastic system through the Chapman-Kolmogorov equation. Generally speaking, the mathematical description of such a system---as codified, for instance, by its L\'{e}vy-Khintchine decomposition---is captured by three distinct contributions: drift, brownian diffusion, and jumps. The standard Langevin equation captured the contributions of drift and Brownian diffusion, while the presence of jumps leads to more exotic stochastic processes---and corresponding mathematical descriptions---through, for instance, L\'{e}vy processes.\footnote{A tractible and familiar example of this in the world of physics would be anomalous diffusion---either super- or sub-diffusive dynamics.} In the absence of jumps, modelling becomes mroe simple, and while the trajectory-level dynamics are goverened by a Langevin equation, the time-dependent evolution of the ensemble-level probability distribution is govenred by a second-order parabolic PDE known as the Fokker-Planck equation.\footnote{In the presence of jumps, the analogous PDE is muchn more difficult to handle, and will often involve fractional derivative terms.}

From a practical standpoint, while trajectory-level dynamics can be relatively simple to understand, and nearly-trivial to simulate numerically, obtaining statistical quantitites can often be numerically expensive and time-consuming, with extra care being necessary to ensure that a sufficiently representative sample of trajectories have been sampled so that the statistical quantities being calculated have convereged.  Put another way, when inferring properties of a stochastic system from trajectory simulations, one has to assume that the entire trajectory distribution has been adequately sampled. This is not, however, a problem for direct solutions to the corresponding PDE, as the solution itself gives all information about the distirbution of states.

Mathematically, the Fokker-Planck equation can be represented by the general flux-conservative form
\begin{equation}
    \partial_t P(x, t) = \partial_x J(x, t)
\end{equation}
where $p(x, t)$ is the space- and time-dependent probability density function of a system, and $J(x, t)$ is the instantaneous probability flux through location $x$ at time $t$.x


% Introduce Fokker-Planck equation as an important equation
The Fokker-Planck equation is a tremendously important relation in the study of all types of stochastic systems. Starting from the most basic `continuity equation' of stochastic processes---the Chapman-Kolmogorov equation---one can show that, under a set of reasonable assumptions on the continuity and smoothness of the process itself, the governing dynamics at the level of a probability distribution is the Fokker-Planck equation.  Put simply, the Fokker-Planck equation takes the form of a 2nd order parabolic partial differential equation, describing the time-dependent evolution of a probability distribution (among other things).
% Superset of other important equations

\part{Numerical Integration}


\section{Numerical integration: general properties}
% Explicit and Implicit integration routine and discussion of general properties of numerical integration
% Introduce general discretization schemes
The central difficulty in solving PDEs is finding an accurate discrete representation of the function and its evolution so that the continuous-space aspects are preserved.  While this may naively seem like a relatively straightforward task, there is a tremendous subtlety to the particular way in which the equations are discretized, with many possible techniques, each having its own benefits and drawbacks.

For the case fo simplicity and concreteness, we will focus on a PDE with one spatial dimension for the time being, and will generalize to multiple spatial dimensions later on. For such a system, the solution to the PDE is given

    \section{Stability}

    \section{Advection}
    % Advection terms

    \section{Diffusion}
    % Diffusion terms

    \section{Fokker-Planck}


    \section{Driven Harmonic Trap}

    \subsection{Excess Work in a Translating Track}

    \subsection{Excess Work in a Breathing Trap}


    \section{Driven Rotary Machine}

    \subsection{Steady-State Flux}

    \subsection{Precision and Driving Accuracy}

    % Other subsections?
    \section{Information Erasure}

    \section{Kolmogorov Backwards Equation}

    \subsection{First passage times}

    \subsection{Financial mathematics: Options pricing}


\part{Steady-state solutions}

    % Work through this section on relaxation methods

\end{document}